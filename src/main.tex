\documentclass[a4paper,11pt,onecolumn,twoside]{article}
% 编译方式:xelatex三次编译
\usepackage{ctex}
\usepackage{fancyhdr}
\usepackage{amsmath,amsfonts,amssymb,graphicx}
\usepackage{subfigure}
\usepackage{indentfirst}
\usepackage{bm}
\usepackage{multicol}
\usepackage{indentfirst}
% \usepackage{picins}
\usepackage{abstract}
\usepackage[T1]{fontenc}
\usepackage{mathptmx}
\usepackage{float}
\usepackage{graphicx}
\usepackage{stfloats}
\usepackage{mdframed}
\usepackage{amsmath}
\usepackage{authblk}
\usepackage{epstopdf}
\usepackage{hyperref}
\usepackage{paralist}
\hypersetup{
    colorlinks=true,
    linkcolor=blue,
    filecolor=blue,
    urlcolor=blue,
    citecolor=black,
}

\addtolength{\topmargin}{-54pt}
\setlength{\oddsidemargin}{-0.9cm}
\setlength{\evensidemargin}{\oddsidemargin}
\setlength{\textwidth}{17.00cm}
\setlength{\textheight}{24.00cm}

\newcounter{TempEqCnt}
\renewcommand{\baselinestretch}{1.1}
\parindent 22pt

\begin{document}
\title{\huge{直肠癌新外科方法}}

\author[]{赵子毅}{}
\author[]{曾树新}{}
\author[]{王美璇}{}
\author[]{陈春霖}{}
\author[]{李易霖}{}
\author[]{季树宇}{}

\affil[]{深圳大学医学部临床医学专业}
\affil[]{\textit {\{20182221138, 20182221138, 20182221138, 20182221138, 20182221138, 2018221138\}@email.szu.edu.cn}}

\renewcommand\Authands{ and }


\date{}


\fancypagestyle{plain}{
    \fancyhf{}
    \rhead{\thepage}
    \chead{\centering{局部解剖学期末论文\\
            \scriptsize{\textbf{Local anatomy}}}}
    \lhead{Jan, 2021}
    \lfoot{}
    \cfoot{}
    \rfoot{}
}

\pagestyle{fancy}
\fancyhf{}
\headheight 26.70523pt
\fancyhead[LE,LO]{Jan, 2021}
\fancyhead[CE,CO]{局部解剖学期末论文}
\fancyhead[RE,RO]{\thepage}
\lfoot{}
\cfoot{}
\rfoot{}

\newenvironment{figurehere}
{\def\@captype{figure}}
{}
\makeatother

\maketitle

\newcommand{\supercite}[1]{\textsuperscript{\cite{#1}}}


\setlength{\oddsidemargin}{1cm}  % 3.17cm - 1 inch
\setlength{\evensidemargin}{\oddsidemargin}
\setlength{\textwidth}{13.50cm}
\vspace{-.8cm}
\begin{center}
    \parbox{\textwidth}{
        \textbf{摘~~~要} \quad  直肠癌
        \\
        \textbf{关键词} \quad 直肠癌,医学}
\end{center}

\vspace{.1cm}
\begin{center}
    \parbox{\textwidth}{
    \begin{center}
        {\large{\textbf{Overview of the application of machine learning in clinical medicine}}}\\[4pt]
        \vspace{-0.5cm}\end{center}
    \begin{center}
        \textbf{Zhao Ziyi}\\[2pt]
        \small{\textit{(Health Science Center of Shenzhen University)}}\\[2pt]
    \end{center}
    {\small{\textbf{Abstract}\quad 直肠癌
        \\
        \textbf{Key Words}\quad 直肠癌, Medicine}}
    }
\end{center}

\setlength{\oddsidemargin}{-.5cm}
\setlength{\evensidemargin}{\oddsidemargin}
\setlength{\textwidth}{17.00cm}

\vspace{0.2cm}
\begin{multicols}{2}

    \section{前言与介绍}
    \section{传统手术方法}
    \section{新手术方法}
    \section{新方法优势与比较}
    \section{总结}

    \small
    \begin{thebibliography}{14}
        \setlength{\parskip}{0pt}  %段落之间的竖直距离
        %%% BIBTEX
    \end{thebibliography}
    \normalsize

\end{multicols}


\clearpage

\end{document}
